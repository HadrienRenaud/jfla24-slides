\documentclass[10pt, dvipsnames]{beamer}

%\setbeameroption{show notes}
%\setbeameroption{show only notes}
%\setbeameroption{show notes on second screen=right}
\usetheme{Boadilla}

\usepackage[T1]{fontenc}
\usepackage[french]{babel}

\usepackage{listings}  % For code
\usepackage{graphicx}  % For figures
\usepackage{caption}  % For figure captions
\usepackage{subcaption}  % For subfigure captions
\usepackage{comment}
\usepackage{tabularx}  % for more controls on tables
\usepackage{adjustbox}
\usepackage{tikz}
\usetikzlibrary{arrows.meta}

\lstset{basicstyle=\ttfamily\scriptsize}

\newcommand\mailto[1]{\href{mailto:#1}{\nolinkurl{#1}}}

\title[Executable semantics of ASL]{Executable semantics of Arm's\\ Architecture Specification Language}
\subtitle{and application to \texttt{CAS}}
\author{Hadrien Renaud}
\institute[UCL]{University College London}
\date{JFLA 2024}

\newtheorem{intuition}{Intuition}

\begin{document}

\begin{frame}
  \titlepage{}
  \note[item]{ASL means Architecture Specification Language}
  \note[item]{ASL is used by Arm to describe the operation of the AArch64 instructions}
  \note[item]{%
    We are going to see a main application of giving semantics to ASL:\
    describing the operation of specific instructions, namely Compare-And-Swap
    CAS}
  \note[item]{For this talk, I want to approach the subject from the Programming Language point of vue}
  \note[item]{We approach the memory model by using Litmus tests}
\end{frame}

\begin{frame}[fragile]{Objectifs}
  \centering
  \begin{tikzpicture}[remember picture, outer sep=4.8pt, >=Stealth]
    \node (instr) at (0, 3) [anchor=south, align=center]
      {\texttt{CAS Xs, Xt, [Xn]}};
    \node (code) at (-1,0) [anchor=east, text width=.37\textwidth, inner ysep=-1em]
      {\lstinputlisting%
        %[frame=shadowbox]%
        %[basicstyle=\ttfamily\tiny]%
        {asl/cas.asl}};
    \node (graph) at (1,0) [anchor=west, text width=.4\textwidth]
      {\includegraphics%
        [ trim={1.3cm, 1.5cm, 2.2cm, 1.5cm}, clip, % left lower right upper
          width=\textwidth, height=\textheight, keepaspectratio]%
        {graphs/old/cas-ok.pdf}};

    \draw[->, thick] (instr) -- (code)
      node [midway, sloped, above, align=center, node font=\small] {Reference\\Manual};
    \draw[->, thick] (instr) -- (graph)
      node [midway, sloped, above, node font=\small] {herd7};
    \draw<1>[<->, very thick, dashed, node font=\huge] (code) -- (graph)
      node [midway] {?};
    \draw<2>[->, thick] (code) -- (graph);
  \end{tikzpicture}
\end{frame}

\begin{comment}
\subsection{LDR}

\begin{frame}{A plain load: \texttt{LDR Xd, [Xn]}}
  Approximativement:
  \begin{enumerate}
    \item Lit l'adresse $x$ depuis le registre \texttt{Xn};
    \item Lit la valeur $v$ depuis la case m\'emoire \`a l'adresse $x$;
    \item \'Ecris la valeur $v$ dans le registre \texttt{Xd}.
  \end{enumerate}
\end{frame}

\begin{frame}[fragile]{A plain load: \texttt{LDR Xd, [Xn]}}
  Simplified ASL code:

  \centering%
  \lstinputlisting{asl/ldr.asl}
\end{frame}
\begin{frame}{A plain load: \texttt{LDR Xd, [Xn]}}
  \texttt{herd7} generated graph:

  \begin{figure}
    \includegraphics%
      [width=\textwidth, height=.9\textheight, keepaspectratio]%
      {graphs/LDR.pdf}
    \caption{\texttt{herd7} generated graph for the instruction \texttt{LDR Xd, [Xn]}}
  \end{figure}
\end{frame}
\end{comment}

\subsection{CAS}

\begin{frame}{Compare-And-Swap \texttt{CAS Xs, Xt, [Xn]} - s\'emantique informelle}
  Approximativement:
  \begin{itemize}
    \item Lire l'adresse $x$ depuis le registre \texttt{Xn}.
    \item Lire la valeur \`a comparer $v_c$ depuis le registre \texttt{Xs}.
    \item Lire la nouvelle valeur $v_n$ depuis le registre \texttt{Xt}.
    \item Lire la valeur $v_m$ dans la case m\'emoire est donn\'ee par l'adresse $x$.
    \item Si $v_m = v_c$, \'ecrire la valeur $v_n$ en m\'emoire \`a l'adresse $x$.
    \item \'Ecrire la valeur $v_m$ dans le registre \texttt{Xs}.
  \end{itemize}

  \pause
  \vfill
  2 cas:
  \begin{description}
    \item[``ok''] si la condition est vraie, et que l'\'ecriture m\'emoire se produit;
    \item[``no''] sinon.
  \end{description}
\end{frame}

\begin{frame}[fragile]{Compare-And-Swap \texttt{CAS Xs, Xt, [Xn]} en ASL}
  Code ASL simplifi\'e :
  \vspace{1em}

  \begin{tabular}{l l}
    \texttt{let address = X[n];}
    & \only<2>{Lire $x$ depuis le registre \texttt{Xn}.}\\
    \texttt{let compare\_value = X[s];}
    & \only<2>{Lire $v_c$ depuis le registre \texttt{Xs}.}\\
    \texttt{let new\_value = X[t];}
    & \only<2>{Lire $v_n$ depuis le registre \texttt{Xt}.}\\
    \texttt{let data = Mem[address];}
    & \only<2>{Lire $v_m$ en m\'emoire \`a l'adresse $x$.}\\
    \texttt{if data == compare\_value then}
    & \only<2>{Si $v_m = v_c$, \ldots}\\
    \texttt{\ \ Mem[address] = new\_value;}
    & \only<2>{\'ecrire $v_n$ en m\'emoire \`a l'address $x$. }\\
    \texttt{end}
    & \only<2>{}\\
    \texttt{X[s] = data;}
    & \only<2>{\'Ecrire $v_m$ dans le registre \texttt{Xs}.}\\
  \end{tabular}

  \note{C'est le code ASL qui est dans le Arm ARM.}
\end{frame}

\begin{frame}{Compare-And-Swap \texttt{CAS Xs, Xt, [Xn]}}
  Graphe g\'en\'er\'e par \texttt{herd7} :

  \begin{figure}
    \includegraphics%
      [width=\textwidth, height=.9\textheight, keepaspectratio]%
      {graphs/old/cas-ok.pdf}
    \caption{Graphe g\'en\'er\'e par \texttt{herd7} pour l'instruction \texttt{CAS X1, X2, [X3]} dans le cas ok}
  \end{figure}
\end{frame}

\begin{frame}[fragile]{Objectifs}
  \centering
  \begin{tikzpicture}[remember picture, outer sep=4.8pt]
    \node (code) at (-1,0) [anchor=east, text width=.37\textwidth]
      {\lstinputlisting%
        %[basicstyle=\ttfamily\tiny]%
        {asl/cas.asl}};
    \node (graph) at (1,0) [anchor=west, text width=.4\textwidth]
      {\includegraphics%
        [ trim={1.3cm, 1.5cm, 2.2cm, 1.5cm}, clip, % left lower right upper
          width=\textwidth, height=\textheight, keepaspectratio]%
        {graphs/old/cas-ok.pdf}};
      \node (instr-text) at (0, -2) [align=center, anchor=north]
        {S\'emantique des instructions AArch64};

    \draw[-Stealth, very thick] (code) -- (graph);
  \end{tikzpicture}
  \note{%
    Arm utilise les deux formats, donc c'est int\'eressant et n\'ecessaire pour Arm de les comparer.
    \\
    contraintes d'ordres => fl\`eches de herd7
    }

\end{frame}

\begin{frame}{D\'ependances d'addresse ou de donn\'ee}
  \begin{columns}[c]
    \begin{column}{.59\textwidth}

      Ordonne une lecture m\'emoire et un autre acc\`es m\'emoire.

      \vspace{1em}
      Inclus des \emph{D\'ependances intra-instruction de donn\'ee}.

      \vspace{2em}
      Exemple:
      \lstinputlisting{graphs/MP+dmb+addr.litmus}
    \end{column}
    \hfill
    \begin{column}{.4\textwidth}
      \centering
      \only<1>{%
        \includegraphics%
          [width=\textwidth, height=.8\textheight, keepaspectratio]%
          {graphs/MP+dmb+addr-basic.pdf}
      }
      \only<2>{%
        \frame{%
        \includegraphics%
          [ trim={30cm, 10cm, 1cm, 1cm}, clip, % left lower right upper 
            width=\textwidth, height=.8\textheight, keepaspectratio]%
          {graphs/MP+dmb+addr-all.pdf}}
      }
    \end{column}
  \end{columns}
\end{frame}

\begin{frame}[fragile]{Objectifs}
  \centering
  \begin{tikzpicture}[remember picture, outer sep=4.8pt, >=Stealth]
    \node (instr) at (0, 3) [anchor=south, align=center]
      {\texttt{CAS Xs, Xt, [Xn]}};
    \node (code) at (-1,0) [anchor=east, text width=.37\textwidth, inner ysep=-1em]
      {\lstinputlisting%
        %[frame=shadowbox]%
        %[basicstyle=\ttfamily\tiny]%
        {asl/cas.asl}};
    \node (graph) at (1,0) [anchor=west, text width=.4\textwidth]
      {\includegraphics%
        [ trim={1.3cm, 1.5cm, 2.2cm, 1.5cm}, clip, % left lower right upper
          width=\textwidth, height=\textheight, keepaspectratio]%
        {graphs/old/cas-ok.pdf}};

    \draw[->, thick] (instr) -- (code)
      node [midway, sloped, above, align=center, node font=\small] {Reference\\Manual};
    \draw[->, thick] (instr) -- (graph)
      node [midway, sloped, above, node font=\small] {herd7};
    \draw[->, thick] (code) -- (graph);
  \end{tikzpicture}
\end{frame}

\begin{frame}{D\'ependances intra-instruction de donn\'ees}
  \begin{intuition}
    Pour $a, b$ deux \'ev\`enements :
    \[
      a \xrightarrow{\texttt{iico\_data}} b \quad \iff \quad b\ \text{utilise des donn\'ees d\'efinies dans}\ a
    \]
  \end{intuition}

  \vfill
  Exemple: \texttt{LDR Xd, [Xn]}

  \begin{columns}[c]
    \hfill
    \begin{column}{.3\textwidth}
      \centering
      \lstinputlisting{asl/ldr.asl}
    \end{column}
    \hfill
    \begin{column}{.3\textwidth}
      \centering
      \includegraphics%
        [width=\textwidth, height=.5\textheight, keepaspectratio]%
        {graphs/LDR.pdf}
    \end{column}
    \hfill
  \end{columns}
\end{frame}

\begin{frame}{D\'ependances intra-instruction de contr\^ole}
  Une autre sorte de d\'ependances intra-instructions!

  \begin{intuition}
    Pour $a, b$ deux \'ev\`enements :
    \[
      a \xrightarrow{\texttt{iico\_ctrl}} b
      \quad \iff \quad
      b\ \text{est dans un \texttt{if} dont la condition utilise}\ a
    \]
  \end{intuition}

  \vfill
  Exemple: \texttt{CSEL Xd, Xn, Xm, cond}

  \begin{columns}[c]
    \begin{column}{.3\textwidth}
      \lstinputlisting{asl/csel.asl}
    \end{column}
    \hfill
    \begin{column}{.3\textwidth}
      \includegraphics%
        [width=\textwidth, height=.7\textheight, keepaspectratio]%
        {graphs/csel.pdf}
    \end{column}
  \end{columns}
\end{frame}

\begin{frame}[fragile]{Objectifs}
  \centering
  \begin{tikzpicture}[remember picture, outer sep=4.8pt, >=Stealth]
    \node (instr) at (0, 3) [anchor=south, align=center]
      {\texttt{CAS Xs, Xt, [Xn]}};
    \node (code) at (-1,0) [anchor=east, text width=.37\textwidth, inner ysep=-1em]
      {\lstinputlisting%
        %[frame=shadowbox]%
        %[basicstyle=\ttfamily\tiny]%
        {asl/cas.asl}};
    \node (graph) at (1,0) [anchor=west, text width=.4\textwidth]
      {\includegraphics%
        [ trim={1.3cm, 1.5cm, 2.2cm, 1.5cm}, clip, % left lower right upper
          width=\textwidth, height=\textheight, keepaspectratio]%
        {graphs/old/cas-ok.pdf}};

    \draw[->, thick] (instr) -- (code)
      node [midway, sloped, above, align=center, node font=\small] {Reference\\Manual};
    \draw[->, thick] (instr) -- (graph)
      node [midway, sloped, above, node font=\small] {herd7};
    \draw[->, thick] (code) -- (graph)
      node [midway, below] {\emph{ASLRef}};
  \end{tikzpicture}

  \note{ASLRef = notre interpreteur, qui construit ces graphes, et que Arm utiliser comme interpreteur officiel.}
\end{frame}

\begin{frame}{Application \`a \texttt{CAS}}
  Code ASL simplifi\'e :
      \lstinputlisting{asl/cas.asl}

  Graphe g\'en\'er\'e :
      \includegraphics%
        [ trim={1.3cm, 1.5cm, 2.2cm, 1.5cm}, clip, % left lower right upper
          width=\textwidth, height=.5\textheight, keepaspectratio]%
        {graphs/asl/cas-ok.pdf}
\end{frame}

\begin{frame}{Oh oh, il y a une divergeance}
  \begin{columns}[T]
    \begin{column}{.49\textwidth}
      Graphe g\'en\'er\'e par herd7:
      \vspace{1em}

      \only<1>{%
        \includegraphics%
          [ trim={1.3cm, 1.5cm, 2.2cm, 1.5cm}, clip, % left lower right upper
            width=\textwidth, height=.9\textheight, keepaspectratio]%
          {graphs/old/cas-ok.pdf}
        }
      \only<2>{%
        \frame{%
        \includegraphics%
          [trim={1cm, 1cm, 8cm, 1.5cm}, clip, % left lower right upper
           width=\textwidth, height=.9\textheight, keepaspectratio]%
          {graphs/old/cas-ok-bolden.pdf}}}
    \end{column}
    \hfill
    \begin{column}{.49\textwidth}
      Graphe g\'en\'er\'e depuis le code ASL :
      \vspace{1em}

      \only<1>{%
        \includegraphics%
          [ trim={1.3cm, 1.5cm, 2.2cm, 1.5cm}, clip, % left lower right upper
            width=\textwidth, height=.9\textheight, keepaspectratio]%
          {graphs/asl/cas-ok.pdf}
        }
      \only<2>{%
        \frame{%
        \includegraphics%
          [trim={10cm, 1cm, 1cm, 1cm}, clip, % left lower right upper
           width=\textwidth, height=.9\textheight, keepaspectratio]%
          {graphs/asl/cas-ok-bolden.pdf}}}
    \end{column}
  \end{columns}
\end{frame}

\begin{frame}{Compare-And-Swap \texttt{CAS Xs, Xt, [Xn]} - s\'emantique informelle}
  Approximativement:
  \begin{itemize}%
      \color{gray}
    \item Lire l'adresse $x$ depuis le registre \texttt{Xn}.
    \item Lire la valeur \`a comparer $v_c$ depuis le registre \texttt{Xs}.
    \item Lire la nouvelle valeur $v_n$ depuis le registre \texttt{Xt}.
    \item Lire la valeur $v_m$ dans la case m\'emoire est donn\'ee par l'adresse $x$.
      \color{black}
    \item Si $v_m = v_c$, \'ecrire la valeur $v_n$ en m\'emoire \`a l'adresse $x$.
    \item \'Ecrire la valeur $v_m$ dans le registre \texttt{Xs}.
  \end{itemize}
  \vfill
  On est ici dans le cas o\`u $v_m = v_c$!
\end{frame}

\begin{frame}[fragile]{Est-ce que c'est grave?}
  On \'ecrit la m\^eme valeur \dots

  \pause
  \vfill
  Ces programmes ont des comportements diff\'erents en fonction de quelle fl\`eche est l\`a :

  \begin{columns}[T]
    \hfill
    \begin{column}{.4\textwidth}
      \lstinputlisting[basicstyle=\tiny\ttfamily]{graphs/MP+rel+CAS-ok-RsRs-addr.litmus}
    \end{column}
    \hfill
    \begin{column}{.4\textwidth}
      \lstinputlisting[basicstyle=\tiny\ttfamily]{graphs/MP+rel+CAS-ok-MRs-addr.litmus}
    \end{column}
    \hfill
    %\begin{column}{.3\textwidth}
      %\lstinputlisting[basicstyle=\tiny\ttfamily]{graphs/MP+rel+CAS-ok-bothRs-addr.litmus}
    %\end{column}
  \end{columns}
\end{frame}

\begin{frame}{\`A qui la faute?}
  ASL ou herd7?

  \pause
  \vspace{1em}
  Les deux sont valides: choix non d\'eterministe entre les deux graphes!

  \vspace{1em}
  Rafiner l'intention des architectes pour la s\'emantique des instructions.
\end{frame}

\begin{frame}{Conclusion}
  \begin{itemize}
    \item Patch\'e dans herd7, discussion en cours dans Arm pour ASL.

    \vfill
    \item Code de l'interpr\'eteur dans
      \url{https://github.com/herd/herdtools7/blob/master/asllib}.

    \vfill
    \item Documents de r\'ef\'erence ASL
      \url{https://developer.arm.com/documentation/DDI0621/00alp0}.

    \vfill
    \item Beaucoup de questions de type-checking, existent aussi au m\^eme endroit
      \url{https://developer.arm.com/documentation/DDI0622/00alp0}.

    \vfill
    \item
        Projet plus vaste men\'e \`a Arm avec:
        \vspace{1em}
        \begin{columns}[T]
          \begin{column}{.3\textwidth}
            \centering
            Jade Alglave \\
            \footnotesize\mailto{j.alglave@ucl.ac.uk} \\
            \footnotesize\mailto{memory-model@arm.com} \\
          \end{column}
          \begin{column}{.3\textwidth}
            \centering
            Luc Maranget \\
            \footnotesize\mailto{Luc.Maranget@inria.fr} \\
          \end{column}
          \begin{column}{.35\textwidth}
            \centering
            et moi \\
            \footnotesize\mailto{hadrien.renaud.22@ucl.ac.uk} \\
            \footnotesize\mailto{hadrien.renaud2@arm.com} \\
          \end{column}
        \end{columns}
  \end{itemize}
\end{frame}

\end{document}
